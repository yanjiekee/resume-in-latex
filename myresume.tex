\documentclass{myresume}

\begin{document}

\section{My Story}

\begin{cvParagraph}
Inside of the broadest cultures are thousands of smaller communities—each with their own cultural vibe that exerts influence on its members. Someone working in a tech startup in the Bay Area is simultaneously living inside of the broad human community, the global Western community, the American community, the U.S. West Coast community, the San Francisco community, the tech industry community, the startup community, the community of their workplace, the community of their college alumni, the community of their extended family, the community of their group of friends, a few other bizarre SF-y situations, and a dozen other communities their particular life happens to be part of (including, if they’re a regular visitor here, the Wait But Why community). Most immediate to each of us are the micro-cultures of our immediate family, closest friends, and romantic relationships. Going against the current of all the larger communities combined tends to be easier than violating the unwritten rules of the most intimate mini-cultures in someone’s life.
\end{cvParagraph}

\end{document}

% Inside of the broadest cultures are thousands of smaller communities—each with their own cultural vibe that exerts influence on its members. Someone working in a tech startup in the Bay Area is simultaneously living inside of the broad human community, the global Western community, the American community, the U.S. West Coast community, the San Francisco community, the tech industry community, the startup community, the community of their workplace, the community of their college alumni, the community of their extended family, the community of their group of friends, a few other bizarre SF-y situations, and a dozen other communities their particular life happens to be part of (including, if they’re a regular visitor here, the Wait But Why community). Most immediate to each of us are the micro-cultures of our immediate family, closest friends, and romantic relationships. Going against the current of all the larger communities combined tends to be easier than violating the unwritten rules of the most intimate mini-cultures in someone’s life.
%
% A culture’s rules, norms, and value systems pertain to a wide spectrum of human experience. A group of friends, for example, has a way they do birthdays, a way they do emojis, a way they do talking behind each other’s backs, a way they do bragging and self-deprecation, a way they do conflict, and so on. They even have a way they do cultural adherence for each area—one group of friends might find it delightful when a certain friend regularly appalls them with their uncharacteristic-for-the-culture bluntness while in another, the same violation might be grounds for dismissal from the community. Some cultures apply pressure to live a certain kind of lifestyle or abide by a particular structure—a culture that shames being single at 30 incentivizes people to be on the lookout for a life partner in their mid-20s, while another one might not apply that pressure at all, driving different behavior.
%
% Living simultaneously in multiple cultures is part of what makes being a human tricky. Do we keep our individual inner values to ourselves and just do our best to match our external behavior to whatever culture we’re currently in a room with? Or do we stay loyal to one particular culture and live by those rules everywhere, even at our social or professional peril? Or do we just go for full authenticity and let our inner values drive our behavior, unaltered, for better or worse? Do we navigate our lives so to seek out external cultures that match our own values and minimize friction? Or do we surround ourselves with a range of conflicting cultures to put some pressure on our inner minds to learn and grow? Whether you consciously realize it or not, you’re making these decisions all the time.
%
% And these decisions matter—because the cultures we spend time have a major influence over us.
%
% This is his mission because the strength of the Higher Mind’s entire being is fed only by truth. It’s a direct correlation: the more access the Higher Mind has to truth, the brighter his light, and the wiser you are.
%
% Given that mission, and the understanding that the mission is incredibly hard and never complete—it’s only rational for the Higher Mind to be entirely humble about his perception of reality at any given moment and totally unattached to the ideas that make up that perception. He sees beliefs as nothing more than the most recent draft of an eternal work in progress, and as he lives more and learns more, nothing makes the Higher Mind happier than a chance to revise that inevitably flawed draft. Because when beliefs are being revised, it’s a signal of progress—of becoming less ignorant, less foolish, less delusional. A change of mind about something is a good sign that his light is getting brighter—and that’s all that matters to the Higher Mind.
%
% And how about the Primitive Mind?
%
% It’s intuitive why the Primitive Mind would object to marital fidelity—but what’s its problem with the Higher Mind’s approach to beliefs? Isn’t truth helpful to its genetic survival mission?
%
% Actually no, it’s not. Truth is mostly irrelevant to the Primitive Mind.
%
% The Primitive Mind’s beliefs are typically installed into its system early on in life, kind of like the way our immune system’s settings are initially configured by our environment. The “intellectual environment” that configures our Primitive Mind’s core beliefs is typically made up of the prevailing beliefs of our family and the broader community we grow up around. On the individual level, the Primitive Mind views those beliefs as a fundamental part of its person’s identity—and therefore about as sacred as the person’s arms or lungs or heart. On the group level, beliefs are the key node that wires its person into a larger giant, which—in the Primitive Mind’s ancient world—means being safe on the lifeboat. For reasons like these, the Primitive Mind puts beliefs into the same ultra-critical category as core biological needs.
%
% Given all of this, the last thing the Primitive Mind wants is for you to feel humble about your beliefs or interested in revising them. It wants you to treat your beliefs as sacred objects—as precious organs in your body or precious seats on a lifeboat. The Primitive Mind treats beliefs like it treats everything else—as nothing more than a means to the singular goal of genetic survival. To the Primitive Mind, the right beliefs are whatever will leave you with the strongest sense of identity and best fuse you with a large, powerful giant. An ever-evolving quest for truth is directly antithetical to these causes.
%
% So when it comes to beliefs, the Primitive Mind doesn’t want truth, it wants confirmation—of your existing beliefs.
%
% So where does this inner conflict leave all of us? As fucking crazy people.
